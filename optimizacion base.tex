Abstract
Mathematics for Economists, a new text for advanced undergraduate and beginning 
graduate students in economics, is a thoroughly modern treatment of the 
mathematics that underlines economics theory.

An abundance of applications to current economic analysis, illustrative 
diagrams, thought-provoking exercises, careful proofs, and a flexible 
organization- these are the advantages that Mathematics for Economists 
brings to today's classroom.

Since optimization plays sucha a major role in economic theory, this chapter on
unsconstrained optimization and the next three chapters on constrained 
optimization can be considered de core of this book. This chapter turns from the
matrix criteria that specify the conditions that caracterize the optima of a 
general differentiable function. Just as techniques of calculus play a major 
role in optimization problems for functions of one variable, they play an 
equally important role for functions of several variables. The main results for
multivariable functions are analogous to the one-dimensional results:

a necessary condition for XXX to be an interior max of XXX is that the first 
derivatives of XXX at XXX be zero, and

if we include an appropriate condition on the second derivatives of XXX, this 
necessary condition becomes a sufficient condition.


\section{Definitions}

The definitions of a maximum and minimum for a function of several variables 
are the same as the corresponding definitions for a function of one variable. 
Let XXX be a real-valued function of XXX variables, whose domain XXX is a 
subset of XXX.



In other words, a point XXX is a local max if there are no nearby points at 
which F takes on a larger value. Of course, a max is always a local max. If we 
want to emphasize that a point XXX is a max of XXX on the whole domain XXX, not 
just a local max, we call XXX a global max or absolute max of XXX on XXX.

To be precise, we should say, for example in (1), that XXX is a maximizer or 
maximum point of XXX, or that XXX has its maximum value at XXX. The word "max" 
is a convenient shortcut. 

Reversing the inequalities in the above four definitions leads to the 
definitions of a global min, a strict global min, a local min, and a strict 
local min, respectively. 

\section{First Order Conditions}

The first order condition for a point XXX to be a max or min of a function XXX
of one variable is that XXX, that is, that XXX be a critical point of XXX. This 
condition requires that XXX not be an endpoint of the interval under 
consideration, in other words, that XXX lie in the interior of the domain of XXX.
The same first order condition works for a function XXX of XXX variables. 
However, a function of XXX variables has XXX first derivatives: the partials 
XXX. The n-dimensional analogue of XXX is that each XXX at XXX. In this case, 
XXX is an interior point of the domain of XXX if there is a whole ball 
XXX about XXX in the domain of XXX. 

Theorem 1 Let XXX be a XXX function defined on a subset XXX of XXX. If XXX is a 
local max or min of XXX in XXX and if XXX is an interior point of XXX, then

equation

Proof We will work with the local max case: the same proof works for the min 
case. Let B = B,(x") be a ball about x• in U with the property that
F(x') 􀄦 F(x) for all x E B. Since x* maximizes F on B, along each line 
segment through x* that lies in B and that is paral1el to one of the axes, 
F takes on its maximum value at x*. In other words, .t7 maximizes the function 
of one variable:

equation

for x1 E (x1* -r, xt + r ). Apply the one-variable maximization criterion of
Theorem 3.3 to each of these none-dimensional problems to conclude that

equation

